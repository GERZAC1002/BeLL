\documentclass[a4paper, twoside, 12pt]{scrreprt}
\usepackage[ngerman]{babel}
\usepackage[utf8]{inputenc}
\usepackage{amsmath}
\usepackage{amssymb}
\usepackage{color}
\usepackage{graphicx}
\usepackage{soul}
\usepackage{natbib}
\title{Bell}
\author{Gernot Zacharias}
\subtitle{Thema: Webanwendung zum Finden eines optimalen Standortes in Abhängigkeit vom Weg mithilfe der Google Maps API}
\begin{document}
\maketitle
\cleardoublepage
\chapter {Inhaltsverzeichnis}

\setcounter{page}{1}
\chapter {Einleitung}

\section {Problemstellung}

\section{Grundlagen}
\subsection{World Wide Web}
Das World Wide Web~\citep{www}
\subsection{HTML}
\subsection{JavaScript}
\subsection{Google Maps}
\subsection{Koordinatenberechnung}
Koordinaten Längengrad und Breitengrad:\\ wgs84 ? \\
https://www.linz.govt.nz/data/geodetic-system/datums-projections-and-heights/geodetic-datums/world-geodetic-system-1984-wgs84 \\
https://www.kompf.de/gps/distcalc.html
\chapter{Existierende Lösungen}
was gibt es für Quellen oder ähnliche Implementierungen
\chapter{Zielstellung}
Gesucht ist eine Anwendung, die diesen kürzesten Gesamtweg (oder diese Zeit) minimiert.
Wenn JavaScript: Interface mit Google Maps, keine Installation nötig, Webanwendungen auf dem Vormarsch…
Wenn andere Programmiersprache, auch da kurz schreiben warum. (z.B. in Schule beigebracht)
\chapter{Algorithmus}
für jedes Ziel weg pro Woche (oder pro Tag…) ausrechnen;
oder Zeit minimieren, es können ja unterschiedliche Fortbewegungsarten sein
erstmal aber Weg minimieren, wenn Zeit ist dann Optionen wie Spritverbrauch, Zeit, eigene Priorität;
erstmal Luftlinie baseline;
optional: kürzester Weg;
Punkte finden, an dem sich das Summe minimiert;
ist das gleich dem Mittelpunkt, wenn man die Strecken entsprechend dem durchschnittlichen Weg verlängert oder kürzt;
für Leipzig oder überall? bei Leipzig könnte man konstanten Umrechnungsfaktor zwischen Längen/Breitengradunterschieden und Metern/Kilometern wählen (kurz sagen wie groß der Fehler ist), ansonsten braucht man Formel;
wenn global, problematisch, wenn Wohnort nicht zugänglich ist;
\chapter{Implementierung}
z.B. Google Maps view-source:http://wohnen-am-ryck.de/karte.php;
Hier werden dann die technischen Details der Implementierung und gegebenenfalls der Algorithmen beschrieben. Verwendete Technologien, Lizenz, Systemvoraussetzungen und so weiter.;
Aufpassen: Fehlerquelle sinus/cosinus Bogenmaß oder Gradmaß;
\chapter{Schlussfolgerung}
\section{Auswertung}
wieviel bringt das wirklich
\section{Ausblick}
Zeit, statt Distanz; nicht Luftlinie, sondern kürzesten Weg (über Google Maps API);global, statt Leipzig; Maximallänge für Wege
\bibliography{bell}{}
\bibliographystyle{natdin}
\end{document}